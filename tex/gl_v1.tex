\documentclass[12pt]{article}

\setlength{\topmargin}{-3pc}
\setlength{\evensidemargin}{-.5pc}
\setlength{\oddsidemargin}{-.5pc}
\setlength{\textwidth}{39pc}
\setlength{\textheight}{55.5pc}
\baselineskip=\normalbaselineskip
\renewcommand{\baselinestretch}{1.2}
\setlength{\parskip}{0.25\baselineskip}

\usepackage{mathrsfs,amsbsy,amssymb,latexsym,amsfonts,amsmath,amsthm,bm,bbold}
\usepackage[nosort]{cite}
\usepackage{hyperref}
\usepackage{tensor}
% \usepackage{algpseudocode}


\newcommand{\sh}{{\mathrm{sh}}}
\newcommand{\ch}{{\mathrm{ch}}}

\newcommand{\htheta}{{\hat\theta}}
\newcommand{\hphi}{{\hat\phi}}
% \newcommand{\th}{{\mathrm{th}}}


\usepackage{bm}
\usepackage{graphicx} %new
\allowdisplaybreaks[1]

\makeatletter
\catcode`\@=11
\@addtoreset{equation}{section}
\renewcommand{\theequation}{\thesection.\arabic{equation}}
\def\@seccntformat#1{\csname the#1\endcsname.~~}
\makeatother

\begin{document}

% \begin{titlepage}
%   \renewcommand{\thefootnote}{\fnsymbol{footnote}}
%   % \begin{flushright}
%   %   preprint****
%   % \end{flushright}
%   \vspace*{1.0cm}

\begin{center}
  {\Large \bf
    RMHMC with embedding $SU(N)$
    in $GL(N)$
  }
  \vspace{1.0cm}

  \centerline{
    Luchang Jin,
    Nobuyuki Matsumoto
    { \today }% %
  %   \footnote{E-mail address: 
  % }
  }

%   % \vskip 0.8cm
%   % {\it
% }
%   \vskip 1.2cm

\end{center}

%   %%%%%%%%%%%%%%%%%%%%%%%%%%%%%%%%%%%%%%% 
%   \begin{abstract}
%     %%%%%%%%%%%%%%%%%%%%%%%%%%%%%%%%%%%%%%%
%     Sinple ising model.
%     %%%%%%%%%%%%%%%%%%%%%%%%%%%%%%%%%%%%%%% 
%   \end{abstract}
%   %%%%%%%%%%%%%%%%%%%%%%%%%%%%%%%%%%%%%%% 
% \end{titlepage}

% \pagestyle{empty}
% \pagestyle{plain}

\tableofcontents
\setcounter{footnote}{0}


\section{Introduction}
\label{sec:intro}

This note tries to spell out possible complexities,
and how they may be overcome them,
in Luchang's idea of
embedding $SU(N)$ in $GL(N)$ for Fourier acceleration.
The original note can be found in
\href{https://rbc.phys.columbia.edu/rbc_ukqcd/individual_postings/luchang/notes/2024-07-11-rmhmc-alg/note-v4.html}{this link}.
This amazing idea was described elegantly already in the original note,
and this note only includes minor modifications and additions of the detailed formulas to it.


\section{Basic idea}
\label{sec:idea}

In RMHMC, we would like to perform the Hamiltonian evolution with the
nonseparable Hamiltonian of the form:
\begin{align}
  H(\pi, U) \equiv \frac{1}{2} \pi_a G_{ab}^{-1}(U) \pi_b + S(U),
\end{align}
where $U \in SU(N)$ and $\pi_a$ the momentum in the Lie algebra.
However, the constraint $U \in SU(N)$
enforces us exponentiate the Lie algebra elements
in the discretized integrators,
which in turn induces nonlinear terms that break the symplecticity
because of the nonseparable nature of the kinetic term.

The idea is to put away the constraint $U \in SU(N)$
and instead work in a larger space $W \in GL(N)$.
We can then take as the dynamical variables the real and imaginary parts
of the matrix elements:
$W \equiv (w_{ij})$, $w_{ij} \equiv x_{ij}+iy_{ij}$,
and use the symplectic integrators
for $x_{ij}, y_{ij}$ with the conventional linear form.

Nevertheless, we want to make the deviation
from $SU(N)$ as small as possible
to make is controllable.
For this purpose, it is convenient to parametrize $W \in GL(N)$
in the form of a polar decomposition:
\begin{align}
  W = e^{i\theta} Q U,
  \label{eq:polar}
\end{align}
where $Q$ is a $N\times N$ hermitian matrix and $U \in SU(N)$.
We will identify the $SU(N)$ part the physical gauge field.
The uncompact nature of $GL(N)$ is then fully described by $Q$,
and therefore we introduce the additional potential
for the $Q\equiv (p_{ij})$ part, such as:
\begin{align}
  S_0(W)
  \equiv \frac{\lambda}{2} \, {\rm Re}\,{\rm tr} \,[W W^\dagger]
  = \frac{\lambda}{2} \,{\rm Re}\,{\rm tr} \,Q^2
  \label{eq:s0}
\end{align}
with the tunable parameter $\lambda \in \mathbb{R}_{>0}$.


\section{Uniqueness and sufficiency of the decomposition}
\label{sec:polar}

To begin with,
we clarify that the decomposition~\eqref{eq:polar}
well determines the gauge field $U$ and
that the determined $U$ spans the entire configuration space.

As is well known, for an invertible matrix $W \in GL(N)$,
the polar decomposition:
\begin{align}
  W = Q V
  \label{eq:polar_VQ}
\end{align}
is unique, where $Q$ is a hermitian matrix and $V \in U(N)$.
Since in the space of $W \in GL(N)$
noninvertible matrices are measure zero,
and furthermore much unlikely in the vicinity of the unitary matrices,
we can assume the decomposition~\eqref{eq:polar_VQ}
always well determines $V$ and $Q$.

The remainder is to decompose $V \in U(N)$
to $e^{i\theta} \in U(1)$ and $U \in SU(N)$.
This is unique only up to the center $\mathbb{Z}_{N}$ of $SU(N)$.
In fact, suppose
\begin{align}
  V = e^{i \theta} U,
\end{align}
then so as
\begin{align}
  V = e^{i \theta'} U'
\end{align}
with
\begin{align}
  \theta' \equiv \theta + 2\pi n/N,
  \quad
  U' \equiv e^{-2\pi i n/N} U
  \quad (n \in \mathbb{Z}).
\end{align}
In other words,
\begin{align}
  U(N) \simeq U(1) \times SU(N) / \mathbb{Z}_{N}.
\end{align}
An obvious way to calculate the representative of the RHS of the isometry is:
\begin{align}
  i\theta \equiv \frac{1}{N} \log \det V,
\end{align}
where we take the principal branch in the logarithm,
and then set:
\begin{align}
  U \equiv e^{-i\theta} V.
\end{align}

Since our action $S(U)$ is not necessarily $\mathbb{Z}_{N}$ symmetric,
we need to carefully choose the integer $m \in \mathbb{Z}_{N}$;
otherwise, the action will have a discontinuous jump in the middle of
the Hamiltonian evolution.
The identification of the appropriate $\mathbb{Z}_{N}$ sector
should be possible as long as
the Hamiltonian evolution well approximates the continuous time evolution,
in which $m$ is unambiguously determined.
For this purpose, in practice, it should be sufficient to trace
the angle variable $\theta$
and to choose $m$ such that the evolution looks most continuous.
In the following, the map~\eqref{eq:polar}
is used under this understanding of the continuity of the Hamiltonian evolution.

Having understood the uniqueness of the decomposition~\eqref{eq:polar},
its surjective property as a map $GL(N) \to U(N)$ is obvious
since it is sufficient to take $W \in U(N) \subset GL(N)$.


\section{Definition of the integration measure in $GL(N)$}
\label{sec:measure}

In order to define the path integral in the larger space
and to prove the equivalence to the original one,
we need to define an appropriate measure for $GL(N)$
and its relation to the Haar measure of $SU(N)$.
As Luchang has correctly pointed out,
the key point is to make it right-invariant under
the multiplication of $U(N)$ matrices,
which ensures, according to the form of the decomposition~\eqref{eq:polar},
the Jacobian associated to
the decomposition of the measure into pieces also right invariant,
meaning that it does not depend on the physical gauge field $U$ (and $\theta$).

We define such a measure by
first defining the right-invariant metric tensor $\textbf{g}$:
\begin{align}
  \textbf{g}
  \equiv
  {\rm tr}\, [dW dW^\dagger].
\end{align}
% where the tensor product is in the space of forms
% and the trace is over the matrix indices.
The one-form $dW$ is defined in the element basis as:
\begin{align}
  dw_{ij} = dx_{ij} + i dy_{ij},
\end{align}
and therefore:
\begin{align}
  \textbf{g}
  =
  \sum_{i,j}[dx_{ij}^2 + dy_{ij}^2].
\end{align}
The associated volume form can be thus derived to be:
\begin{align}
  (dW)
  \equiv
  \prod_{i,j}
  \big(
  dx_{ij} dy_{ij}
  \big).
  \label{eq:measure_flat}
\end{align}

In $SU(N)$, the convenient one-form basis
is given by the Maurer-Cartan form:
\begin{align}
  \Theta \equiv i T_a \Theta_a = dU U^{-1}. \label{eq:MaurerCartan}
\end{align}
The generators $T_a$ are here taken to be hermitian and normalized as:
\begin{align}
  {\rm tr}\,[T_a T_b] = \frac{1}{2} \delta_{ab}.
\end{align}
$\Theta_a$ is dual to the right-invariant derivative $D_a$:
\begin{align}
  D_a U = i T_a U,
\end{align}
and thus:
\begin{align}
  dU = \Theta_a D_a U,
\end{align}
in accordance to eq.~\eqref{eq:MaurerCartan}.
We define the Haar measure as:
\begin{align}
  (dU) \equiv \prod_a \Theta_a.
\end{align}

With the generators,
hermitian matrices can be expanded as:
\begin{align}
  dQ = q_0 \textbf{1}_N + q_a T_a.
\end{align}
The map~\eqref{eq:polar} thus induces the relation:
\begin{align}
  dW
  =
  i e^{i\theta} Q U d\theta
  +
  e^{i\theta} U dq_0
  +
  e^{i\theta} T_a U dq_a
  +
  i e^{i\theta} Q T_a U \Theta_a.
  \label{eq:expand_dw}
\end{align}
The metric can be written as:
\begin{align}
  \textbf{g}
  =
  {\rm tr}\, Q^2  d\theta^2
  +
  2 \, {\rm tr}\,[T_a Q^2] d\theta \, \Theta_a
  +
  dq_0^2
  +
  \frac{1}{2} dq_a^2
  +
  {\rm tr}\, [T_a T_b Q^2]  \Theta_a \Theta_b.
\end{align}
Thus, as advertised above,
the decomposition of the measure can be done with the
Jacobian that only depends on $Q$:
\begin{align}
  (dW)
  =
  \sqrt{\det g(Q)}
  \,
  d\theta
  \,
  dq_0
  \,
  \Big(\prod_a dq_a\Big)
  \Big(\prod_a \Theta_a \Big),
  \label{eq:measure_w}
\end{align}
where the determinant factor is for the matrix:
\begin{align}
  g(Q) =
  \left(
  \begin{array}{c c c c}
    {\rm tr}\, Q^2 & {\rm tr}\,[T_b Q^2] & & \\
    {\rm tr}\,[T_a Q^2] & (1/2){\rm tr}\,[\{T_a, T_b\} Q^2] & & \\
                   && 1 & \\
                   && & (1/2)\textbf{1}_{N^2-1}
  \end{array}
  \right).
\end{align}


\section{Path integral in the larger space}
\label{sec:path_integral}

It is now easy to relate the original path integral:
\begin{align}
  Z_{SU(N)}
  \equiv
  \int(dU)\,
  e^{-S(U)}
\end{align}
to the path integral in the larger space:
\begin{align}
  Z_{GL(N)}
  \equiv
  \int (dW)\,
  e^{ - S(U)- S_0(W)},
  \label{eq:large_path_int}
\end{align}
where $U$ is regarded as a function of $W$ through the map~\eqref{eq:polar}.
Indeed, plugging the expressions~\eqref{eq:s0} and \eqref{eq:measure_w}
into eq.~\eqref{eq:large_path_int}, we find:
\begin{align}
  Z_{GL(N)}
  &=
    (2\pi)
    \int
    dq_0
    \,
    \Big(\prod_a dq_a\Big)
    \sqrt{\det g(Q)}
    e^{-S_0(Q)}
    \cdot
    \int
    (dU)\, e^{-S(U)} \nonumber \\
  &=
    Z_{SU(N)}
    \cdot
    (2\pi)
    \int
    dq_0
    \,
    \Big(\prod_a dq_a\Big)
    \sqrt{\det g(Q)}
    e^{-S_0(Q)}.
\end{align}
In other words,
the two partition functions are related by
an integral over hermitian matrices.

The above means that we can calculate the expectation values of
observables ${\mathcal O}(U)$ in the larger path integral
and the nontrivial jacobian factor will automatically drop out
to ensure them to be the same as those of the original physical system.



\section{Jacobian matrix and the RMHMC Hamiltonian}
\label{sec:rmhmc_hamil}

Having established the equivalence,
we can focus on writing down the
RMHMC algorithm taking
$w_{ij}=x_{ij}+iy_{ij}$ as dynamical variables.
In this section, we describe a simple
choice of the Hamiltonian for the RMHMC.

The basis of the tangent vectors
that is conjugate to $\{dx_{ij}, dy_{ij}\}$
is $\{\partial_{x_{ij}}, \partial_{y_{ij}} \}$.
Accordingly, we can expand the momentum $\pi$ in the form:
\begin{align}
  \pi
  \equiv
  \pi^x_{ij}\partial_{x_{ij}}
  +
  \pi^y_{ij}\partial_{y_{ij}}.
\end{align}
We have another one-form basis
$\{d\theta, \Theta_a, dq_0, dq_a\}$,
which has the conjugate basis
$\{\partial_\theta, D_a, \partial_{q_0}, \partial_{q_a} \}$:
\begin{align}
  \pi
  \equiv
  \pi_\theta \partial_{\theta}
  +
  \pi_a D_a
  +
  p_0 \partial_{q_0}
  +
  p_a \partial_{q_a}.
\end{align}
$\pi_a$ is the momentum conjugate to the physical variables
that we would like to accelerate.

To obtain the relation between the two expansion coefficients,
let us rewrite the relation~\eqref{eq:expand_dw}
in the matrix form:
\begin{align}
  &\left[
    \begin{array}{c c}
      dx_{ij}
      &
        dy_{ij}
    \end{array}
    \right]
    =
    \left[
    \begin{array}{c c c c}
      d\theta & \Theta_a & dq_0 & dq_a
    \end{array}
    \right]
    \times
    \nonumber\\
  &~~~~~~~~~
    \times
    \left[
    \begin{array}{c c}
      -\sin\theta\, {\rm Re}\,(QU)_{ij}-\cos\theta\,{\rm Im}\,(QU)_{ij}
      &
        \cos\theta\, {\rm Re}\,(QU)_{ij}-\sin\theta\,{\rm Im}\,(QU)_{ij}
      \\
      -\sin\theta\, {\rm Re}\,(QT_aU)_{ij}-\cos\theta\,{\rm Im}\,(QT_aU)_{ij}
      &
        \cos\theta\, {\rm Re}\,(QT_aU)_{ij}-\sin\theta\,{\rm Im}\,(QT_aU)_{ij}
      \\
      \cos\theta\, {\rm Re}\,(QU)_{ij}-\sin\theta\,{\rm Im}\,(QU)_{ij}
      &
        \sin\theta\, {\rm Re}\,U_{ij}+\cos\theta\,{\rm Im}\,U_{ij}
      \\
      \cos\theta\, {\rm Re}\,(T_aU)_{ij}-\sin\theta\,{\rm Im}\,(T_aU)_{ij}
      &
        \cos\theta\, {\rm Re}\,(T_aU)_{ij}+\cos\theta\,{\rm Im}\,(T_aU)_{ij}
    \end{array}
    \right]
    \nonumber\\
  &
    ~~~~~~~~~~~~~~~~~~
    \equiv
    \left[
    \begin{array}{c c c c}
      d\theta & dq_0 & dq_a & \Theta_a
    \end{array}
    \right]
    J(\theta, U, Q).
    \label{eq:transf}
\end{align}
We then have:
\begin{align}
  \left[
  \begin{array}{c}
    \partial_{x_{ij}}
    \\
    \partial_{y_{ij}}
  \end{array}
  \right]
  =
  J(\theta, U, Q)^{-1}
  \left[
  \begin{array}{c}
    \partial_\theta
    \\
    D_a
    \\
    \partial_{q_0}
    \\
    \partial_{q_a}
  \end{array}
  \right].
\end{align}

Though the kinetic term for the unphysical sector
can be taken arbitrarily,
for simplicity, we take the Hamiltonian:
\begin{align}
  H_{GL(N)}
  \equiv
  \frac{1}{2 \mu_\theta^2}
  \pi_\theta^2
  +
  \frac{1}{2}
  \pi_a G^{-1}_{ab} (U) \pi_b
  +
  \frac{1}{2 \mu_0^2}
  p_0^2
  +
  \frac{1}{2 \mu^2}
  p_a^2
  +
  S(U)
  +
  S_0(Q)
\end{align}
with the tunable parameters $\mu_\theta, \mu_0, \mu$.
The jacobian matrix $J(\theta, U, Q)$ mixes the
physical and unphysical modes according to
the transformation law~\eqref{eq:transf}.
We write the corresponding expression in the $x_{ij}, y_{ij}$
basis simply as:
\begin{align}
  H_{GL(N)}
  (\pi, W)
  =
  \frac{1}{2} \pi_I \tilde{G}_{IJ}^{-1}(W) \pi_J + S(U) + S_0(W),
  \label{eq:Hamil_GLN}
\end{align}
where the index $I$ runs all the $2N^2$ elements
$(w_I) \equiv ( x_{ij}, y_{ij} )$.


\section{RMHMC in $GL(N)$}
\label{sec:rmhmc_gln}

We now have
a path integral~\eqref{eq:large_path_int}
over $2N^2$ real variables
$(w_I) = (x_{ij}, y_{ij})$
with the flat measure~\eqref{eq:measure_flat}.
We are thus ready to write down the
RMHMC with the Hamiltonian~\eqref{eq:Hamil_GLN}.
We show below the algorithm with the simple implicit leapfrog:
\begin{enumerate}
\item Suppose we have a configuration $W$.
\item Generate $\pi_I$
  with the distribution:
  \begin{align}
    \propto e^{-(1/2) \pi_I \tilde G_{IJ}^{-1}(W) \pi_J}.
  \end{align}
\item
  Integrate the leapfrog
  multiple times to get a new phase space configuration $(\pi',W')$:
  \begin{align}
    \pi_I^{1/2}
    &=
      \pi_I
      -
      \frac{\epsilon}{2}
      \partial_{q_I} H(\pi_{1/2}, W)
      ,
      \label{eq:p_half}
    \\
    w_I^{1/2}
    &=
      w_I
      +
      \frac{\epsilon}{2}
      \partial_{p_I} H(\pi_{1/2}, W)
      ,
      \label{eq:q_half}
    \\
    w_I^{\prime}
    &=
      w_I^{1/2}
      +
      \frac{\epsilon}{2}
      \partial_{\pi_I} H(\pi_{1/2}, W')
      ,
      \label{eq:q_prime}
    \\
    \pi'_I
    &= \pi_I^{1/2}
      -
      \frac{\epsilon}{2}
      \partial_{w_I} H(\pi_{1/2}, W').
      \label{eq:p_prime}
  \end{align}
\item Accept/reject with the probability:
  \begin{align}
    \min \Big(
    1, e^{-H_{GL(N)}(\pi',W') + H_{GL(N)}(\pi, W)}
    \Big).
  \end{align}
\end{enumerate}




\end{document}
