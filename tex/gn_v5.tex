\documentclass[12pt]{article}

\setlength{\topmargin}{-3pc}
\setlength{\evensidemargin}{-.5pc}
\setlength{\oddsidemargin}{-.5pc}
\setlength{\textwidth}{39pc}
\setlength{\textheight}{55.5pc}
\baselineskip=\normalbaselineskip
\renewcommand{\baselinestretch}{1.2}
\setlength{\parskip}{0.25\baselineskip}

\usepackage{mathrsfs,amsbsy,amssymb,latexsym,amsfonts,amsmath,amsthm,bm,bbold}
\usepackage[nosort]{cite}
\usepackage{hyperref}
\usepackage{tensor}
% \usepackage{algpseudocode}


\newcommand{\sh}{{\mathrm{sh}}}
\newcommand{\ch}{{\mathrm{ch}}}

\newcommand{\htheta}{{\hat\theta}}
\newcommand{\hphi}{{\hat\phi}}
% \newcommand{\th}{{\mathrm{th}}}


\usepackage{bm}
\usepackage{graphicx} %new
\allowdisplaybreaks[1]

\makeatletter
\catcode`\@=11
\@addtoreset{equation}{section}
\renewcommand{\theequation}{\thesection.\arabic{equation}}
\def\@seccntformat#1{\csname the#1\endcsname.~~}
\makeatother

\begin{document}

% \begin{titlepage}
%   \renewcommand{\thefootnote}{\fnsymbol{footnote}}
%   % \begin{flushright}
%   %   preprint****
%   % \end{flushright}
%   \vspace*{1.0cm}

\begin{center}
  {\Large \bf
    RMHMC by embedding $SU(N)$
    into $M_N(\mathbb{C})$
  }
  \vspace{1.0cm}

  \centerline{
    Luchang Jin,
    Nobuyuki Matsumoto
    { \today }% %
  %   \footnote{E-mail address: 
  % }
  }

%   % \vskip 0.8cm
%   % {\it
% }
%   \vskip 1.2cm

\end{center}

%   %%%%%%%%%%%%%%%%%%%%%%%%%%%%%%%%%%%%%%% 
%   \begin{abstract}
%     %%%%%%%%%%%%%%%%%%%%%%%%%%%%%%%%%%%%%%%
%     Sinple ising model.
%     %%%%%%%%%%%%%%%%%%%%%%%%%%%%%%%%%%%%%%% 
%   \end{abstract}
%   %%%%%%%%%%%%%%%%%%%%%%%%%%%%%%%%%%%%%%% 
% \end{titlepage}

% \pagestyle{empty}
% \pagestyle{plain}

\tableofcontents
\setcounter{footnote}{0}


\section{Introduction}
\label{sec:intro}

This note tries to spell out possible complexities,
and how they may be overcome,
in Luchang's idea of
embedding $SU(N)$ in $M_N(\mathbb{C})$,
the space of $N\times N$ complex matrices,
for RMHMC/Fourier acceleration.
The original note can be found in
\href{https://rbc.phys.columbia.edu/rbc_ukqcd/individual_postings/luchang/notes/2024-07-11-rmhmc-alg/note-v4.html}{this link}.
This amazing idea was described elegantly already in the original note,
and this note only includes minor modifications and additions of the detailed formulas to it.


\section{Basic idea}
\label{sec:idea}

In RMHMC, we would like to perform the Hamiltonian evolution with the
nonseparable Hamiltonian of the form:
\begin{align}
  H(\pi, U) \equiv \frac{1}{2} \pi_a G_{ab}^{-1}(U) \pi_b + S(U)
  - \frac{1}{2} \log \det G(U),
\end{align}
where $U \in SU(N)$ and $\pi_a$ the momentum in the Lie algebra.
However, the constraint $U \in SU(N)$
enforces us to exponentiate the Lie algebra elements
in the discretized integrators,
which induces nonlinear terms that break symplecticity
because of the nonseparable nature of the kinetic term.

The idea is to put away the constraint $U \in SU(N)$
and instead work in the larger space $W \in M_N(\mathbb{C})$.
We can then take as the dynamical variables the real and imaginary parts
of the matrix elements:
$W \equiv (w_{jk})$, $w_{jk} \equiv x_{jk}+iy_{jk}$,
and use the symplectic integrators
for $x_{jk}, y_{jk}$ with the conventional linear form.

Nevertheless, we want to make the deviation
from $SU(N)$ as small as possible
to make it controllable.
For this purpose, it is convenient to parametrize $W \in M_N(\mathbb{C})$
in the form of a polar decomposition:
\begin{align}
  W = e^{i\theta} \Phi U,
  \label{eq:polar}
\end{align}
where $\Phi$ is an $N\times N$ hermitian matrix and $U \in SU(N)$.
We will identify the $SU(N)$ part the physical gauge field.
The uncompact nature of $M_N(\mathbb{C})$ is then fully described by $\Phi$,
and therefore we introduce the additional potential
for $\Phi$ such as:
\begin{align}
  S_0(W)
  \equiv \frac{\lambda}{2} \, {\rm tr} \,[W W^\dagger-\mathbb{1}_N]
  = \frac{\lambda}{2} \,{\rm tr} \,
  \big[
  (\Phi-\mathbb{1}_N)(\Phi-\mathbb{1}_N)^\dagger
  \big]
  \label{eq:s0}
\end{align}
with the tunable parameter $\lambda \in \mathbb{R}_{>0}$.


\section{Uniqueness and sufficiency of the decomposition}
\label{sec:polar}

To begin with,
we clarify that the decomposition~\eqref{eq:polar}
well-determines the gauge field $U$ and
that the determined $U$ spans the entire configuration space
of the original theory.

As is well known, for an invertible matrix $W \in GL(N)$,
the polar decomposition:
\begin{align}
  W = \Phi \Omega
  \label{eq:polar_VQ}
\end{align}
is unique, where $\Phi$ is a hermitian matrix and $\Omega \in U(N)$.
Since in the space of $W \in M_N(\mathbb{C})$
the matrices with $\det W=0$ are measure zero,
and furthermore since $\det W=0$ will not
happen in the vicinity of the unitary matrix,
we may safely assume below
that the decomposition~\eqref{eq:polar_VQ}
well-determines $\Phi$ and $\Omega$
even for $W \in M_N(\mathbb{C})$
in our practical situation.

The remainder is to decompose $\Omega \in U(N)$
into $e^{i\theta} \in U(1)$ and $U \in SU(N)$.
This is unique up to the center $\mathbb{Z}_{N}$ of $SU(N)$.
In fact, suppose
\begin{align}
  \Omega = e^{i \theta} U,
\end{align}
then so as
\begin{align}
  \Omega = e^{i \theta'} U'
\end{align}
with
\begin{align}
  \theta' \equiv \theta + 2\pi n/N,
  \quad
  U' \equiv e^{-2\pi i n/N} U
  \quad (n \in \mathbb{Z}).
\end{align}
In other words,
\begin{align}
  U(N) \simeq U(1) \times SU(N) / \mathbb{Z}_{N}.
\end{align}
An obvious way to calculate the representative of the RHS of the isometry is:
\begin{align}
  i\theta \equiv \frac{1}{N} \log \det \Omega,
\end{align}
where we take the principal branch in the logarithm,
and then set:
\begin{align}
  U \equiv e^{-i\theta} \Omega.
\end{align}

Since our action $S(U)$ is not necessarily $\mathbb{Z}_{N}$ symmetric,
we need to carefully choose the integer $n \in \{0,\cdots,N-1\}$;
otherwise, the action can have a discontinuous jump
as we step over to a different $\mathbb{Z}_N$ sector.
The identification of the appropriate $\mathbb{Z}_{N}$ sector
should be possible when
the discretized evolution well-approximates the continuous time limit,
in which $n$ is unambiguously determined by smoothness.
In practice, it should be sufficient to trace
the angle variable $\theta$
and to choose $n$ such that the evolution looks most continuous.
In the following, the map~\eqref{eq:polar}
is used under this understanding of the continuity of the Hamiltonian evolution,
which makes the decomposition unique.

Having understood the uniqueness,
its surjective property as a map $M_N(\mathbb{C}) \to SU(N)$ is obvious
since it is sufficient to take $W \in SU(N) \subset M_N(\mathbb{C})$.


\section{Definition of the integration measure in $M_N(\mathbb{C})$}
\label{sec:measure}

In order to define the path integral in the larger space
and to prove the equivalence to the original one,
we first define an appropriate measure for $M_N(\mathbb{C})$
and its relation to the Haar measure of $SU(N)$.
As Luchang correctly pointed out,
the key point is to make it right-invariant under
the multiplication of $U(N)$ matrices,
which ensures, according to the form of the decomposition~\eqref{eq:polar},
% the Jacobian associated to
the associated decomposition of the measure to be also right invariant,
meaning that it does not depend on the physical gauge field $U$
(and also $\theta$).

We define such a measure by
first defining the right-invariant metric tensor $\textbf{g}$
(which turns out to be also left-invariant):
\begin{align}
  \textbf{g}
  \equiv
  {\rm tr}\, [dW dW^\dagger].
\end{align}
% where the tensor product is in the space of forms
% and the trace is over the matrix indices.
The one-form $dW\equiv (dw_{jk})$ is defined in the element basis as:
\begin{align}
  dw_{jk} = dx_{jk} + i dy_{jk},
\end{align}
and therefore:
\begin{align}
  \textbf{g}
  =
  \sum_{j,k}[dx_{jk}^2 + dy_{jk}^2].
\end{align}
The associated volume form can be thus derived to be:
\begin{align}
  (dW)
  \equiv
  \prod_{j,k}
  \big(
  dx_{jk} dy_{jk}
  \big).
  \label{eq:measure_flat}
\end{align}

In $SU(N)$, a convenient one-form basis
is given by the Maurer-Cartan form:
\begin{align}
  \Theta \equiv i T_a \Theta_a = dU U^{-1}. \label{eq:MaurerCartan}
\end{align}
The generators $T_a$ of $SU(N)$ are here taken to be hermitian and normalized as:
\begin{align}
  {\rm tr}\,[T_a T_b] = \delta_{ab}.
\end{align}
$\Theta_a$ is dual to the right-invariant derivative $D_a$:
\begin{align}
  D_a U = i T_a U,
\end{align}
{\it i.e.},
\begin{align}
  dU = \Theta_a D_a U,
\end{align}
in accordance to eq.~\eqref{eq:MaurerCartan}.
The Haar measure is:
\begin{align}
  (dU) \equiv \prod_a \Theta_a
\end{align}
up to an irrelevant normalization constant.

With the traceless hermitian generators $T_a$,
hermitian matrices can be expanded as:
\begin{align}
  \Phi = \phi_0 \mathbb{1}_N + \phi_a T_a.
\end{align}
The map~\eqref{eq:polar} then induces the relation:
\begin{align}
  dW
  =
  i e^{i\theta} \Phi U d\theta
  +
  e^{i\theta} U d\phi_0
  +
  e^{i\theta} T_a U d\phi_a
  +
  i e^{i\theta} \Phi T_a U \Theta_a.
  \label{eq:expand_dw}
\end{align}
The metric can be rewritten accordingly:
\begin{align}
  \textbf{g}
  =
  {\rm tr}\, \Phi^2  d\theta^2
  +
  2 \, {\rm tr}\,[T_a \Phi^2] d\theta \, \Theta_a
  +
  d\phi_0^2
  +
  d\phi_a^2
  +
  {\rm tr}\, [T_a T_b \Phi^2]  \Theta_a \Theta_b.
\end{align}
Thus, as advertised above,
the decomposition of the measure can be done with the
Jacobian that only depends on $\Phi$:
\begin{align}
  (dW)
  =
  \sqrt{\det g(\Phi)}
  \,
  d\theta
  \,
  d\phi_0
  \,
  \Big(\prod_a d\phi_a\Big)
  \Big(\prod_a \Theta_a \Big),
  \label{eq:measure_w}
\end{align}
where the determinant is
given by:
\begin{align}
  \det\,g(\Phi)
  & =
    \det\,\left(
    \begin{array}{c c c c}
      {\rm tr}\, \Phi^2 & {\rm tr}\,[T_b \Phi^2] & & \\
      {\rm tr}\,[T_a \Phi^2] & (1/2){\rm tr}\,[\{T_a, T_b\} \Phi^2] & & \\
                        && 1 & \\
                        && & \mathbb{1}_{N^2-1}
    \end{array}
    \right) \\
  & =
    \det\,\left(
    \begin{array}{c c}
      {\rm tr}\, \Phi^2 & {\rm tr}\,[T_b \Phi^2] \\
      {\rm tr}\,[T_a \Phi^2] & (1/2){\rm tr}\,[\{T_a, T_b\} \Phi^2]
    \end{array}
    \right).
\end{align}


\section{Path integral in the larger space}
\label{sec:path_integral}

It is now easy to relate the original path integral:
\begin{align}
  Z_{SU(N)}
  \equiv
  \int(dU)\,
  e^{-S(U)}
\end{align}
to the path integral in the larger space:
\begin{align}
  Z_{M_N(\mathbb{C})}
  \equiv
  \int (dW)\,
  e^{ - S(U)- S_0(W)},
  \label{eq:large_path_int}
\end{align}
where $U$ is regarded as a function of $W$ through the map~\eqref{eq:polar}.
Indeed, plugging the expressions~\eqref{eq:s0} and \eqref{eq:measure_w}
into eq.~\eqref{eq:large_path_int}, we find:
\begin{align}
  Z_{M_N(\mathbb{C})}
  &=
    (2\pi)
    \int
    d\phi_0
    \,
    \Big(\prod_a d\phi_a\Big)
    \sqrt{\det g(\Phi)}
    e^{-S_0(\Phi)}
    \cdot
    \int
    (dU)\, e^{-S(U)} \nonumber \\
  &=
    Z_{SU(N)}
    \cdot
    (2\pi)
    \int
    d\phi_0
    \,
    \Big(\prod_a d\phi_a\Big)
    \sqrt{\det g(\Phi)}
    e^{-S_0(\Phi)}.
\end{align}
In other words,
the two partition functions are related by
an integral over hermitian matrices
that is factorized.
The factorized integral is a Gaussian integral
over a function that is $O(\phi^{N^2})$
and thus has a finite value.

The above shows that we can calculate the expectation values of
observables ${\mathcal O}(U)$ in the larger path integral
and the nontrivial Jacobian factor will automatically drop out
to ensure them to be the same as those of the original physical system.



\section{Jacobian matrix and the RMHMC Hamiltonian}
\label{sec:rmhmc_hamil}

Having established the equivalence,
we can focus on writing down the
RMHMC algorithm by taking
$w_{jk}=x_{jk}+iy_{jk}$ as dynamical variables.
In this section, we describe a simple
choice of the Hamiltonian for the RMHMC.

The basis of the tangent vectors
that is conjugate to $\{dx_{jk}, dy_{jk}\}$
is $\{\partial_{x_{jk}}, \partial_{y_{jk}} \}$.
Accordingly, we can expand the momentum $\pi$ in the form:
\begin{align}
  \pi
  \equiv
  \pi^x_{jk}\partial_{x_{jk}}
  +
  \pi^y_{jk}\partial_{y_{jk}}.
\end{align}
We have another one-form basis
$\{d\theta, \Theta_a, d\phi_0, d\phi_a\}$,
which has the conjugate basis
$\{\partial_\theta, D_a, \partial_{\phi_0}, \partial_{\phi_a} \}$:
\begin{align}
  \pi
  \equiv
  \pi_\theta \partial_{\theta}
  +
  \pi_a D_a
  +
  \rho_0 \partial_{\phi_0}
  +
  \rho_a \partial_{\phi_a}.
\end{align}
$\pi_a$ is the momentum conjugate to the physical variables
that we would like to accelerate.

To obtain the relation between the two expansion coefficients,
let us rewrite the relation~\eqref{eq:expand_dw}
in the matrix form:
\begin{align}
  &\left[
    \begin{array}{c c}
      dx_{jk}
      &
        dy_{jk}
    \end{array}
    \right]
    =
    \left[
    \begin{array}{c c c c}
      d\theta & \Theta_a & d\phi_0 & d\phi_a
    \end{array}
    \right]
    \times
    \nonumber\\
  &~~~~~~~~~
    \times
    \left[
    \begin{array}{c c}
      -\sin\theta\, {\rm Re}\,(\Phi U)_{jk}-\cos\theta\,{\rm Im}\,(\Phi U)_{jk}
      &
        \cos\theta\, {\rm Re}\,(\Phi U)_{jk}-\sin\theta\,{\rm Im}\,(\Phi U)_{jk}
      \\
      -\sin\theta\, {\rm Re}\,(\Phi T_aU)_{jk}-\cos\theta\,{\rm Im}\,(\Phi T_aU)_{jk}
      &
        \cos\theta\, {\rm Re}\,(\Phi T_aU)_{jk}-\sin\theta\,{\rm Im}\,(\Phi T_aU)_{jk}
      \\
      \cos\theta\, {\rm Re}\,(\Phi U)_{jk}-\sin\theta\,{\rm Im}\,(\Phi U)_{jk}
      &
        \sin\theta\, {\rm Re}\,U_{jk}+\cos\theta\,{\rm Im}\,U_{jk}
      \\
      \cos\theta\, {\rm Re}\,(T_aU)_{jk}-\sin\theta\,{\rm Im}\,(T_aU)_{jk}
      &
        \cos\theta\, {\rm Re}\,(T_aU)_{jk}+\cos\theta\,{\rm Im}\,(T_aU)_{jk}
    \end{array}
    \right]
    \nonumber\\
  &
    ~~~~~~~~~~~~~~~~~~
    \equiv
    \left[
    \begin{array}{c c c c}
      d\theta & d\phi_0 & d\phi_a & \Theta_a
    \end{array}
    \right]
    J(\theta, U, \Phi).
    \label{eq:transf}
\end{align}
We then have:
\begin{align}
  \left[
  \begin{array}{c}
    \partial_{x_{jk}}
    \\
    \partial_{y_{jk}}
  \end{array}
  \right]
  =
  J(\theta, U, \Phi)^{-1}
  \left[
  \begin{array}{c}
    \partial_\theta
    \\
    D_a
    \\
    \partial_{\phi_0}
    \\
    \partial_{\phi_a}
  \end{array}
  \right].
\end{align}

Though the kinetic term for the unphysical sector
can be taken arbitrarily,
for simplicity, we take the Hamiltonian:
\begin{align}
  H_{M_N(\mathbb{C})}
  \equiv
  \frac{1}{2 \mu_\theta^2}
  \pi_\theta^2
  +
  \frac{1}{2}
  \pi_a G^{-1}_{ab} (U) \pi_b
  +
  \frac{1}{2 \mu^2}
  (
  \rho_0^2
  +
  \rho_a^2
  )
  +
  S(U)
  +
  S_0(\Phi)
  - \frac{1}{2} \log \det G(U),
\end{align}
with the tunable parameters $\mu_\theta$ and $\mu$.
The Jacobian matrix $J(\theta, U, \Phi)$ mixes the
physical and unphysical modes according to
the transformation law~\eqref{eq:transf}.
We write the corresponding expression in the $x_{jk}, y_{jk}$
basis simply as:
\begin{align}
  H_{M_N(\mathbb{C})}
  (\pi, W)
  =
  \frac{1}{2} \pi_I \tilde{G}_{JK}^{-1}(W) \pi_J + S(U) + S_0(W)
  - \frac{1}{2} \log \det \tilde G(U),
  \label{eq:Hamil_GLN}
\end{align}
where we absorbed the normalization constants into the determinant
and the index $I$ runs all the $2N^2$ elements
$(w_I) \equiv ( x_{jk}, y_{jk} )$.


\section{RMHMC in $M_N(\mathbb{C})$}
\label{sec:rmhmc_gln}

We now have
a path integral~\eqref{eq:large_path_int}
over $2N^2$ real variables
$(w_I) = (x_{jk}, y_{jk})$
with the flat measure~\eqref{eq:measure_flat}.
We are thus ready to write down the
RMHMC with the Hamiltonian~\eqref{eq:Hamil_GLN}.
We show below the algorithm with the simple implicit leapfrog:
\begin{enumerate}
\item Suppose we have a configuration $W$.
\item Generate $\pi_I$
  with the distribution:
  \begin{align}
    w_{\rm init}(\pi; W)
    \equiv
    \sqrt{ \frac{ {\rm det}\, \tilde G(W) }{(2\pi)^{2N^2}} }
    \times
    e^{-(1/2) \pi_I \tilde G_{IJ}^{-1}(W) \pi_J},
  \end{align}
  which is normalized against the flat measure:
\begin{align}
  (d\pi) \equiv \prod_I d\pi_I.
\end{align}
\item
  Integrate the leapfrog
  multiple times to get a new phase space configuration $(\pi',W')$:
  \begin{align}
    \pi_I^{1/2}
    &=
      \pi_I
      -
      \frac{\epsilon}{2}
      \partial_{w_I} H(\pi^{1/2}, W)
      ,
      \label{eq:p_half}
    \\
    w_I^{1/2}
    &=
      w_I
      +
      \frac{\epsilon}{2}
      \partial_{\pi_I} H(\pi^{1/2}, W)
      ,
      \label{eq:q_half}
    \\
    w_I^{\prime}
    &=
      w_I^{1/2}
      +
      \frac{\epsilon}{2}
      \partial_{\pi_I} H(\pi^{1/2}, W')
      ,
      \label{eq:q_prime}
    \\
    \pi'_I
    &= \pi_I^{1/2}
      -
      \frac{\epsilon}{2}
      \partial_{w_I} H(\pi^{1/2}, W').
      \label{eq:p_prime}
  \end{align}
\item Accept/reject with the probability:
  \begin{align}
    \min \Big(
    1, e^{-H_{M_N(\mathbb{C})}(\pi',W') + H_{M_N(\mathbb{C})}(\pi, W)}
    \Big).
  \end{align}
\end{enumerate}

This algorithm satisfies the detailed balance
for the equilibrium distribution:
\begin{align}
  w_{\rm eq}(W)
  \equiv
  \frac{1}{Z_{M_N(\mathbb{C})}} \times e^{-S(U)-S_0(W)}.
\end{align}
Indeed, the probability of getting the configuration $W'$ accepted
from $W$ is:
\begin{align}
  P_{\rm accept}(W' \leftarrow W)
  &\equiv
    \int(d\pi')(d\pi)\,
    \min \Big(
    1, e^{-H_{M_N(\mathbb{C})}(\pi',W') + H_{M_N(\mathbb{C})}(\pi, W)}
    \Big)\times \\
  & ~~~~~~~~~~~~~~
    \times
    \delta\Big(
    (\pi', W'), \Psi_{\rm MD} (\pi,W)
    \Big)
    w_{\rm init}(\pi;W),
\end{align}
where $\Psi_{\rm MD} (\pi,W)$ stands for the
outcome of the symplectic integrator.
As usual, using that
\begin{align}
  w_{\rm init}(\pi; W) w_{\rm eq}(W)
  =
  \frac{1}{(2\pi)^{N^2}Z_{M_N(\mathbb{C})}}
  e^{-H_{M_N(\mathbb{C})}(\pi, W)},
\end{align}
we have:
\begin{align}
  &
  P_{\rm accept}(W' \leftarrow W)w_{\rm eq}(W) \nonumber\\
  &=
    \frac{1}{(2\pi)^{N^2}Z_{M_N(\mathbb{C})}}
    \int(d\pi')(d\pi)\,
    \min \Big(
    e^{-H_{M_N(\mathbb{C})}(\pi',W')},
    e^{-H_{M_N(\mathbb{C})}(\pi, W)}
    \Big)\times
    \delta\Big(
    (\pi', W'), \Psi_{\rm MD} (\pi,W)
    \Big) \nonumber \\
  &=
    \frac{1}{(2\pi)^{N^2}Z_{M_N(\mathbb{C})}}
    \int(d\pi')(d\pi)\,
    \min \Big(
    e^{-H_{M_N(\mathbb{C})}(\pi',W')},
    e^{-H_{M_N(\mathbb{C})}(\pi, W)}
    \Big)\times
    \delta\Big(
    (-\pi,W), \Psi_{\rm MD}(-\pi', W')
    \Big) \nonumber \\
  &=
    \frac{1}{(2\pi)^{N^2}Z_{M_N(\mathbb{C})}}
    \int(d\pi)(d\pi')\,
    \min \Big(
    e^{-H_{M_N(\mathbb{C})}(\pi, W)},
    e^{-H_{M_N(\mathbb{C})}(\pi',W')}
    \Big)\times
    \delta\Big(
    (\pi,W), \Psi_{\rm MD}(\pi', W')
    \Big) \nonumber \\
  &
    =
    P_{\rm accept}(W \leftarrow W') w_{\rm eq}(W').
\end{align}





\end{document}
